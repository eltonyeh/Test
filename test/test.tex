\documentclass[math]{answer}

\begin{document}
\helloworld{104}{高雄市立高雄高級中學}{科學班}{數學}

\section{填充題test}

\begin{questions}
    \question 
    \answer{$9$}{1}{
        外角為$\ang{180}$減去內角,因此此凸$n$邊形的外角度數也會形成等差數列,其中最大角為$\ang{64}$,最小角為$\ang{16}$,而外角和必為$\ang{360}$,可知此等差數列有$9$項,因此$n=9$。
    }
    
    \question 
    \answer{$2\sqrt{5}$}{1.5}{
        由內分比,我們可以知道
        \[
            \overline{AB}:\overline{AC}=\overline{DB}:\overline{DC}=5:3.
        \]
        因為$\angle C$為直角,因此此三角形為$3-4-5$的直角三角形,再得到$\overline{AB}=10$、$\overline{AC}=6$。因為$I$為內心,因此由內分比性質可以得到
        \[
            \overline{AI}:\overline{ID}=\overline{AB}:\overline{BD}=10:5=2:1.
        \]
        我們接著只要算出$\overline{AD}$即可求得答案,而$\overline{AD}=\sqrt{{\overline{AC}}^2+{\overline{CD}}^2}=3\sqrt{5}$,可得所求為$2\sqrt{5}$。
    }
    
    \question 
    \answer{$a>c>b$}{0.5}{
        這題根本送分題,應該可以直接看的出來。注意到
        \[
            501\times 2001=(1001-500)(1001+1000)=1001\times 1001+500\times 1001-500\times 1000=1001^2+500.
        \]
        利用以上的式子就可以不需要將其乘開即可知道誰大誰小(就算真的沒想到,直接乘開也不難,送分題)。結果會是$a>c>b$。
    }

    \question 
    \answer{$-16$}{1}{
        將$(1+2x)(1+ax)(1+bx^2)$展開可以得到
        \[
            2abx^4+(ab+2b)x^3+(2a+b)x^2+(a+2)x+1=dx^4+cx^3+1.
        \]
        比較$x^2$項、$x$項係數可以得到$a=-2$、$b=4$,再得$d=2ab=-16$。
    }
    
    \question 
    \answer{$\ang{390}$}{2}{
        我們令$\overline{DE}$與$\overline{AB}$、$\overline{AG}$分別交於$P$、$Q$兩點,由外角定理可以知道
        \[
            \angle A=\angle EQG-\angle APQ.
        \]
        我們不難發現,$\angle EQG$為中間的凸七邊形的其中一個內角,而$\angle APQ$為中間凸七邊形的外角,我們其實可以將$\angle A$到$\angle G$全換成凸七邊形的內角減外角,而將$\angle A$到$\angle G$全部加起來即為七邊形的內角和減去外角和,故所求為
        \[
            (\ang{900}-\ang{360})-\angle A-\angle C=\ang{390}.
        \]
    }
    
    \question 
    \answer{$70$}{2.5}{
        考慮將$\overline{AD}$及$\overline{BC}$延長交於$P$點,則因為$\overline{EF}:\overline{AB}=1:2$、$\overline{CE}:\overline{EB}=2:3$,可以求出
        \[
            \overline{PC}:\overline{CE}:\overline{EB}=1:2:3.
        \]
        又因為相似三角形的面積與邊長平方成比例,可得梯形$ABEF$與$ABCD$的面積關係為
        \[
            ABEF\mbox{面積}:ABCD\mbox{面積}=(6^2-3^2):(6^2-1^2).
        \]
        故所求即為$54\times \frac{35}{27}=70$。
    }
    
    \question 
    \answer{$\frac{9}{2}$}{1}{
        將原式移項平方可得
        \[
            4x-2=1+2x+2\sqrt{2x}.
        \]
        將根號移至同一邊並再次平方,化簡後可得到
        \[
            4x^2-20x+9=0.
        \]
        因式分解得$(2x-9)(2x-1)=0$,得到$x$的可能解有$\frac{9}{2}$和$\frac{1}{2}$兩個,帶回檢查發現$\frac{1}{2}$不合。
    }
    
    \question 
    \answer{$3$}{2.5}{
        因為$180$為偶數,可知$\sqrt{x}$和$\sqrt{x+180}$必為同奇偶,令$\sqrt{x}=n$,可知$\sqrt{x+180}$必為$n+2k$,其中$k$為正整數。由以上假設,我們知道
        \[
            180=(n+2k)^2-n^2=4nk+4k^2\Rightarrow 45=k(n+k).
        \]
        可以發現,$k$最大值為$6$,否則右式會超過左式。且$k$要是$45$的因數,可得$k$只能為$1$ or $3$ or $5$,此時$x=44^2$ or $12^2$ or $4^2$,共三組解。
    }
    
    \question 
    \answer{$-5$}{3.5}{
        首先,我們將方程式寫成
        \[
            ||x-1|-2|=-a\pm 3.
        \]
        注意以上是兩個方程式,而不是一個。對於任一個方程式,最多只會有$2\times 2=4$個解(因為左邊是兩層絕對值)。然而,由題目可以知道這兩個方程式的解總共要有4個,所以兩個方程式都要有解,不管是三個、兩個還是一個(其實不可能只有一個,讀者可以思考看看為甚麼)。於是,可以得到$-a-3, -a+3\geq 0$,即$-3\geq a$。
        
        再來,我們可以得到
        \[
            |x-1| = 2 \pm (-a \pm 3).
        \]
        這個方程式在$a$值不同時,可能會有$2, 1, 0$組解。我們將所有情況寫出來,得到
        \[
            |x-1| = 
            \begin{cases}
                5-a \\ -1-a \\ 5+a \\ -1+a
            \end{cases}
        \]
        四個方程式。由題目,可以知道這四個方程式的解全部共有5個。這代表說四個方程式中,會有一個是恰好有一根的,所以$5-a, -1-a, 5+a, -1+a$中會有一個為0。因此,$a=5, 1, -1, -5$。又$a\leq -3$,我們得到$a=-5$,代回成立。
    }
    
    \question 
    \answer{$384$}{2.5}{
        令此半圓的半徑為$r$,過$O$作$\overline{OH}$垂直$\overline{AE}$於$H$,$H$即為$\overline{AE}$與半圓的切點,我們可以列出以下的關係式
        \[
            \sqrt{\overline{AD}^2+\overline{DE}^2}=\overline{AE}=\overline{AH}+\overline{HE}=\sqrt{\overline{AO}^2-r^2}+\sqrt{\overline{OE}^2-r^2}.
        \]
        而其中又有
        \[
        \begin{cases}
            \overline{AD}=2r \\
            \overline{DE}=\overline{CD}-\overline{CE}=\sqrt{\overline{AO}^2-r^2}-\sqrt{\overline{OE}^2-r^2}.
        \end{cases}
        \]
        將這兩個長度代入上式並平方可以得到
        \[
            \begin{split}
                4r^2&={\left (\sqrt{\overline{AO}^2-r^2}+\sqrt{\overline{OE}^2-r^2}\right )}^2-{\left (\sqrt{\overline{AO}^2-r^2}-\sqrt{\overline{OE}^2-r^2}\right )}^2\\
                &=4\left (\sqrt{\overline{AO}^2-r^2}\right )\left (\sqrt{\overline{OE}^2-r^2}\right ).
            \end{split}
        \]
        再將$\overline{AO}=20$、$\overline{OE}=15$代入,得到
        \[
            r^2=\sqrt{(400-r^2)(225-r^2)}.
        \]
        再平方一次,可解得$r=12$。故所求長方形面積為$2r\times \sqrt{\overline{AO}^2-r^2}=24\times 16=384$。
    }
    
    \question 
    \answer{$\frac{1}{9}$}{2.5}{
        以公式解可得題目中方程式的兩解為
        \[
            \frac{1\pm \sqrt{1-8a}}{2a}.
        \]
        首先,我們觀察比較大的解,即$\frac{1+\sqrt{1-8a}}{2a}$,這個式子的分子必為正數,而原方程的兩根皆為正整數,因此分母也要是正的,可得到$a>0$。接著,我們觀察另一個解,即$\frac{1-\sqrt{1-8a}}{2a}$,注意到我們有以下的不等式
        \[
            0<\frac{1-\sqrt{1-8a}}{2a}<\frac{1-(1-8a)}{2a}=4.
        \]
        第一個不等號成立是因為兩根皆為正整數,而第二個不等號成立是因為$a>0$,因此$1-8a$一定是$0\sim 1$的數,再得$\sqrt{1-8a}>1-8a$,由此可得到第二個不等號。由以上不等號,我們可以得到原方程的其中一個解一定是$1$ or $2$ or $3$,分別將這三個數代入原式,求出$a$再檢查,可以得到只有$3$符合題意,此時$a=\frac{1}{9}$,兩個解分別為$3$和$6$。
    }
    
    \question 
    \answer{$4$}{2.5}{
        首先,由題目給的式子可以得到$a+b=d+e$,因此這四個數一定是$2,3,4,5$ or $2,3,5,6$ or $3,4,5,6$,也就是$c=6$ or $4$ or $2$,我們分為這三種情況討論:
        \begin{enumerate}
            \item 當$c=6$時,$a+b=d+e=7$,而$b+c+d$一定超過$8$,因此此情況不可能有解。
            \item 當$c=4$時,$a+b=d+e=8$,故$b+d=5$,可得$b+d$只能是$2+3$,共有$2$種可能。
            \item 當$c=2$時,$a+b=d+e=9$,故$b+d=8$,可得$b+d$只能是$3+5$,共有$2$種可能。
        \end{enumerate}
        綜合以上情況,得到所求共有$4$組。
    }
\end{questions}

\section{計算證明題}

\begin{questions}
    \question
    \sprout{見詳解}
    \panswernl{2}{
        我們知道圓心角等於兩倍圓周角,因為$\triangle ABC$為正三角形,利用角度與圓周的互換可得到
        \[
            2\angle E=\arc{AB}-\arc{PC}=\arc{BC}-\arc{PC}=\arc{BP}=2\angle BCP.
        \]
        由上可以得到$\angle E=\angle BCP$,同理可得$\angle F=\angle CBP$,因此$\triangle BCE\sim \triangle FBC$。由相似三角形邊長比例關係即可得到所求。
    }
    
    \question
    \sprout{見詳解}
    \panswernl{3}{
        首先,可以知道若青蛙回到原點,則它往北移動的次數一定跟往南移動的次數一樣,因此往東和往西的次數最多差$1$(最後一次移動),而往東一次會移動$2$單位,往西一次$1$單位,若要讓往東的距離與往西的距離相同,那麼往西一定要比往東多次,而且因為兩者次數最多差$1$,因此只能是往東一次、往西兩次,由此可得$n$必等於$5$。
    }
\end{questions}
\end{document}
